%%%%%%%%%%%%%%%%%%%%%%%%%%%%%%%%%%%%%%%%%
% Medium Length Professional CV
% LaTeX Template
% Version 2.0 (8/5/13)
%
% This template has been downloaded from:
% http://www.LaTeXTemplates.com
%
% Original author:
% Trey Hunner (http://www.treyhunner.com/)
%
% Important note:
% This template requires the resume.cls file to be in the same directory as the
% .tex file. The resume.cls file provides the resume style used for structuring the
% document.
%
%%%%%%%%%%%%%%%%%%%%%%%%%%%%%%%%%%%%%%%%%

%----------------------------------------------------------------------------------------
%	PACKAGES AND OTHER DOCUMENT CONFIGURATIONS
%----------------------------------------------------------------------------------------

\documentclass{resume} % Use the custom resume.cls style

\usepackage[left=0.75in,top=0.6in,right=0.75in,bottom=0.6in]{geometry} % Document margins
\usepackage[colorlinks,
            linkcolor=red,
            anchorcolor=blue,
            citecolor=green
            ]{hyperref}
\usepackage{verbatim}
\usepackage{enumitem} 
\usepackage{xcolor}
\hypersetup{%
unicode=true, CJKbookmarks=true, bookmarksnumbered=true,
bookmarksopen=true, bookmarksopenlevel=1, breaklinks=true,
colorlinks=false, plainpages=false, pdfpagelabels, pdfborder=0 0 0 }
\name{Xuejian Wang} % Your name
\address{(+1) 412-251-1130 \\ {5506 Wilkins Ave, Pittsburgh, PA, 15217} \\ \href{mailto:xuejianw@andrew.cmu.edu}{xuejianw@andrew.cmu.edu}}

\begin{document}
%\begin{CJK*}{GBK}{song}
\vspace{-1em}
%----------------------------------------------------------------------------------------
%	EDUCATION SECTION
%----------------------------------------------------------------------------------------

\vspace{20pt}


\begin{rSection}{Education}
%{\bf }%\hfill { } \\
\textbf{Carnegie Mellon University} \hfill \textbf{Pittsburgh, USA}
\vspace{-5pt}
\item[・]Joint PhD student in Machine Learning and Public Policy \hfill Sep. 2018 - Present
\vspace{-5pt}
\item[・]Advisor: Prof. Leman Akoglu \\ \\
\textbf{Shanghai Jiao Tong University (SJTU)} \hfill \textbf{Shanghai, China}\\ % \smallskip 
B.S in Information Security \hfill Sep. 2014 - Jun. 2018
%\vspace{-5pt} 
%\item[・] GPA: 81.7(Freshman), 83.9(Sophomore), 85.2(Junior)
\vspace{-5pt}
\item[・] Research Assistant, APEX Data \& Knowledge Management Lab
\vspace{-5pt}
\item[・] Advisor: Prof. \href{http://wnzhang.net}{Weinan Zhang}, Prof. Yong Yu and Prof. Jun Wang(University College London)
\end{rSection}

\begin{rSection}{RESEARCH INTERESTS}
My research interest lie in deep learning and representation learning, as well as their applications in recommender systems, natural language processing and anomaly detection. For more information, please view \href{xuejianwang.com}{xuejianwang.com}.
\end{rSection}

%----------------------------------------------------------------------------------------
%	PUBLICATION SECTION
%----------------------------------------------------------------------------------------

\begin{rSection}{Honors $\&$ Awards}
\begin{rSubsection}{}{}{}{}
\item[] \textbf{CMU Presidential Fellowship} \hfill{2018 - 2019}
\item[] \textbf{SJTU Outstanding Graduate} \hfill{2018}
\item[] \textbf{KDD Travel Award} \hfill{2017}
\item[] \textbf{Rongchang Science and Technology Innovation Scholarship (Nomination)} \hfill{2017}
\item[] SJTU Excellent Scholarship \hfill{2017$\&$2016}
\item[] SJTU Excellent Student Award \textbf{(Top 5\%)}  \hfill{2017$\&$2016}
\item[] Second Prize, China Undergraduate Mathematical Contest in Modeling 2016, Shanghai \hfill{2016}
\end{rSubsection}
\end{rSection}

\begin{rSection}{Publications}
%{\bf }%\hfill { } \\
\begin{rSubsection}{Large-scale Interactive Recommendation with Tree-structured Policy Gradient}{}{}{}
\item Haokun Chen, Xinyi Dai, Weinan Zhang, Han Cai, \textbf{Xuejian Wang}, Ruiming Tang, Yuzhou Zhang, Yong Yu
\item In \emph{Proceedings of the 33rd AAAI Conference on Artificial Intelligence (AAAI-19)}. \textbf{AAAI, 2019}
\end{rSubsection}
\vspace{-2pt}
\begin{rSubsection}{Neural Link Prediction over Aligned Networks}{}{}{}
\item Xuezhi Cao, Haokun Chen, \textbf{Xuejian Wang}, Weinan Zhang, and Yong Yu.
\item In \emph{Proceedings of the 32nd AAAI Conference on Artificial Intelligence (AAAI-18)}. \textbf{AAAI, 2018}
\end{rSubsection}
\vspace{-2pt}
\begin{rSubsection}{Dynamic Attention Deep Model for Article Recommendation by Learning Human Editors' Demonstration}{}{}{}
\item \textbf{Xuejian Wang}*, Lantao Yu*, Kan Ren, Guanyu Tao, Weinan Zhang, Yong Yu, Jun Wang.
\item In \emph{Proceedings of the 23rd ACM SIGKDD International Conference on Knowledge Discovery and Data Mining}. \textbf{KDD 2017}
\end{rSubsection}
\end{rSection}
%----------------------------------------------------------------------------------------
%	RESEARCH EXPERIENCE SECTION
%----------------------------------------------------------------------------------------
%\vspace{0.1em}

\begin{rSection}{Research Experiences}
\begin{rSubsection}{Detecting Unseen Risk Class in Online Textual Data}{Sep. 2018 - Present}{Advisor: Prof. Leman Akoglu}{DATA Lab, CMU}
\item Detecting emerging risk class from large corpus of text on the Internet, mainly news and twitters
\item Generalizing algorithm to other similar datasets. Still ongoing
\end{rSubsection}
\newpage
\begin{rSubsection}{Large-scale Interactive Recommendation via Reinforcement Learning}{Dec. 2017 - April. 2018}{Advisor: Prof. Weinan Zhang}{APEX Data $\&$ Knowledge Management Lab, SJTU}
\item This study focuses on large discrete action space problem in reinforcement learning based recommender systems
\item Employing a Tree-structured Policy Gradient Recommendation (TPGR) framework to accelerate sampling
\end{rSubsection}
\begin{rSubsection}{Neural Link Prediction over Aligned Networks}{Aug. 2017 - Sep. 2017}{Advisor: Prof. Yong Yu}{APEX Data $\&$ Knowledge Management Lab, SJTU}
\item Implemented \emph{LINE} in Tensorflow for comparison and tuned parameters to best performance
\item Revised the whole paper and contributed over 100 submits 
\item Surveyed papers about social networks and proposed attention based framework which we left as future work
\end{rSubsection}

\begin{rSubsection}{Dynamic Attention Deep Model for Article Recommendation
by Learning Human Editors’ Demonstration}{Oct. 2016 - Feb. 2017}{Advisor: Prof. Weinan Zhang}{APEX Data $\&$ Knowledge Management Lab, SJTU}
\item Built a text classification network to model the editors' underlying criterion varied with many factors such as time, current affairs, etc., for a famous Chinese media website
\item Employed attention mechanism to address data drift problem, resulting in more robust and stable predictions
\item Proposed a Dynamic Attention Deep Model (DADM) which outperformed other baselines in an A/B test
\item Our paper was accepted to KDD 2017 and the proposed DADM model was utilized in practical cases, automating the quality article selection process to alleviate the editors' working load\\\\
\end{rSubsection}
\end{rSection}
\vspace{-20pt}
%----------------------------------------------------------------------------------------
%	INTERNSHIP
%----------------------------------------------------------------------------------------
\begin{rSection}{Professional Activities}
\textbf{External Reviewer} \hfill WWW Journal
\end{rSection}


\begin{rSection}{Internship Experience}
\begin{rSubsection}{ULU Technologies Inc.}{Nov. 2016 - Feb. 2017}{R$\&$D  Engineer Intern}{}
\item Developed a practical algorithm for article recommendation which is used in production 
\item Improved coding ability, learned how to independently conduct experiments and developed communication skills
%\item Developed test cases and wrote general design documents for an industrial reliability testing platform.
%\item Programmed Oracle database in Java for a new testing module to update data and improve efficiency.
\end{rSubsection}
\end{rSection}


%----------------------------------------------------------------------------------------
%	TECHNICAL STRENGTHS SECTION
%----------------------------------------------------------------------------------------




\begin{rSection}{Skills}
{\bf Machine Learning: }
\hspace*{3.0 cm} Tensorflow(primary), Pytorch, XGBoost, Sklearn, Keras\\
{\bf Programming Languages: }
\hspace*{1.8 cm} Python(primary), MATLAB, C++, R, Verilog and \LaTeX  \\
\end{rSection}


\clearpage
%\end{CJK*}
\end{document}